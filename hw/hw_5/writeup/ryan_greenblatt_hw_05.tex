\documentclass{article}
\usepackage[utf8]{inputenc}
\usepackage{amsmath}
\usepackage{amssymb}
\usepackage{amssymb}
\usepackage{upgreek}
\usepackage[colorlinks = true, linkcolor = black, urlcolor  = blue]{hyperref}

\usepackage[margin=1.5in]{geometry}
\usepackage{relsize}
\usepackage{color}
\usepackage{graphicx}
\usepackage{subcaption}
\usepackage{longtable}

\title{APMA 2822b Homework 4} 
\author{Ryan Greenblatt}
\date{April 2019}

\begin{document}

\setlength\parindent{0pt}

\renewcommand{\thesubsection}{\alph{subsection}}

\maketitle

\section{}

Code is attached in the email. I have included a CMakeLists.txt file which can
be used to compile the code on the CCV. The \verb cuda/10.0.130  module must be
loaded. I also added the SLURM scripts I used. Note that gencode is specified
to be "arch=compute\_70,code=sm\_70" which may cause issues on older GPUs.

\subsection*{Algorithm and Implementation}

I found the maximum on the CPU and GPU by traversing the array and 
comparing each element to the maximum element found thus far. On the CPU,
a single thread was dedicated to each array while parallel reductions were
used on the GPU. Two GPU implementations were tested for this task. 
One used one warp per row and assigned multiple elements to each
thread initially. Shuffle operations were used to determine the maximum
among the thread maximums found. The other implementation another has 
one thread per every element and uses shuffle operations
for parallel reduction and shared memory to collect data from each warp. 
The reductions performed after syncing with shared memory are also performed
using shuffle operations.  \\

I found the nth maximum on the CPU by using the quickselect algorithm which
has expected $O(n)$ runtime. The algorithm involves selecting some element
in the array and using that element to partition the array into two parts.
The first part is less than the element while the second part is greater than
or equal to the element. This can be done entirely in place. Then, the nth
element must be in the sub array of smaller elements if nth is less than the
number of elements in the smaller array. In the other case, it must be in
the greater or equal sub array. This can be recursively done until the 
nth element is clear. Because this algorithm can't be easily parallelized,
a different algorithm was used on the GPU. \\

Least significant digit radix sort was used on the GPU. This has $O(n)$
runtime. This involves sorting an array of elements by some number of the least
significant bits. I used 2 bits. The values can then be binned and sorted based
on those bits by computing how many values occur before each value in the
array. This can be done by counting the number of values of each bin per thread
and using prefix sum parallel reductions. The number of threads used was the
size of the array divided by 4 to have 4 values per thread for most threads.
Because the number of values per warp is less than 256, the counts per each
warp may be stored in 8 bit unsigned integers. Because there are 4 bins,
these values can be packed into a 32 bit unsigned integer. Instead of looping 
over each bin in the warp local reductions, packed 32 bit unsigned integers
can be operated on to compute prefix sums. After the warp local operations
32 bit unsigned integers must be used to ensure no overflow occurs.

\section{Results}

All testing was done on the CCV with a GPU (V100) and 8 CPU cores.  Time
required to copy data and convert between formats wasn't counted. I ran each
test 10 times and averaged the last 8 runs to time required for shared memory
migration. \\

The GPU implementation for finding the maximum element was significantly faster
than the CPU implementation. The version which uses one warp for every array
rather than a thread for every value was slightly faster. \\

The GPU radix sort was significantly slower than the CPU algorithm.
 
On the GPU, the cuSPARSE implementation was the fastest.  The ELLPACK and
standard implementations were very similar and almost as fast as the cuSPARSE
implementation. On the CPU, the CRS implementation was the fastest. The optimal
value for ELLPACK row loop unrolling was found to be 4.  Storing the number of
elements in each row and only looping through the required elements improved
CPU and GPU performance.  Using a texture resulted in no improvement for CRS or
ELLPACK.  The timing results can be seen in tables 1 and 2. \\

Unified and device memory had similar performance after the first iteration.
The first iteration using managed memory which switched from the CPU to GPU was
substantially slower because the managed memory needed to be transfered from
device to host. \\

Interestingly, the average kernel timings reported by nvprof were substantially
different from the timing found by the code. I used the CUDA event API to time
CUDA kernels in the code. I tested on both my laptop and the CCV and the
disparity wasn't at all present on my laptop. The average kernels durations as
reported by nvprof are given in table 3 below. I think that most likely the
kernel timings reported by nvprof are incorrect and the issue may be related to
the nvprof error which using CUDA 10. The issue may also be occurring in CUDA
9, but isn't properly reported.  \\ 

Profiling the code using nvprof allowed for detailed inspection of the time
required for each CUDA API call. The timings for each CUDA API call are given
in table 4 below. \texttt{cudaFree} was found to use a large percentage of the
overall time of the program (437.68 ms). It isn't clear to me why this would be
the case. \texttt{cudaMalloc}, \texttt{cudaMallocManaged}, and
\texttt{cudaMallocPitch} also took substantial program execution time.
\texttt{cudaMemcpy} consumed most of the remaining time spent on CUDA API
calls. \\


\begin{table}[] 
  \centering 
  \begin{tabular}{|l|l|l|l|l|} 
    \hline Method               & Average time & 1            & 2            & 3 \\
    \hline CPU                      & 1.600510e-04 & 7.989560e-04 & 1.593470e-04 & 1.586510e-04 \\
    \hline GPU                      & 4.646400e-05 & 5.334400e-05 & 4.668800e-05 & 4.742400e-05 \\
    \hline GPU texture memory     & 4.848800e-05 & 1.091520e-04 & 5.129600e-05 & 4.912000e-05 \\
    \hline CPU managed before GPU & 1.610724e-04 & 2.447670e-04 & 2.344980e-04 & 1.753500e-04 \\
    \hline GPU managed            & 4.674400e-05 & 4.469568e-03 & 4.918400e-05 & 4.796800e-05 \\
    \hline CPU managed after GPU  & 1.610724e-04 & 2.447670e-04 & 2.344980e-04 & 1.753500e-04 \\
    \hline cuSPARSE                 & 3.886400e-05 & 1.065280e-04 & 4.323200e-05 & 3.977600e-05 \\ \hline
  \end{tabular} 
  \caption{The recorded timings for each method utilizing the CRS
    data format. The average is the average over the 10 runs excluding the first
    two runs. Only the first 3 runs are shown to highlight transfer times
  associated with managed memory.} 
\end{table}

\begin{table}[] 
  \centering 
  \begin{tabular}{|l|l|l|l|l|} 
    \hline Method                 & Average time & 1            & 2            & 3 \\
    \hline CPU                    & 2.352189e-04 & 7.101640e-04 & 3.498100e-04 & 2.545160e-04 \\
    \hline GPU                    & 4.766800e-05 & 5.331200e-05 & 4.723200e-05 & 4.729600e-05 \\
    \hline GPU texture memory     & 4.765600e-05 & 4.881280e-04 & 4.928000e-05 & 4.761600e-05 \\
    \hline CPU managed before GPU & 2.949605e-04 & 1.024591e-03 & 8.099850e-04 & 3.533850e-04 \\
    \hline GPU managed            & 4.756000e-05 & 8.975392e-03 & 5.948800e-05 & 4.678400e-05 \\
    \hline CPU managed after GPU  & 2.949605e-04 & 1.024591e-03 & 8.099850e-04 & 3.533850e-04 \\ \hline
  \end{tabular}

  \caption{The recorded timings for each method utilizing the ELLPACK data
    format.  The average is the average over the 10 runs excluding the first two
    runs.  Only the first 3 runs are shown to highlight transfer times associated
  with managed memory.} 
\end{table}

\begin{table}[] 
  \centering 
  \begin{tabular}{|l|l|} 
    \hline Method                 & Average kernel time \\
    \hline ELLPACK                & 865.07us \\
    \hline CRS                    & 260.40us \\
    \hline ELLPACK texture memory & 82.038us \\
    \hline CRS texture memory     & 43.023us \\
    \hline cuSPARSE               & 38.012us \\ \hline
  \end{tabular}
  \caption{The average running time of each
    kernel as reported by nvprof. I think these
  timings are likely wrong.}
\end{table}

\begin{table}[] 
  \centering 
  \begin{tabular}{|l|l|} 
    \hline API call           & Time \\
    \hline cudaFree           & 437.68ms \\
    \hline cudaMalloc         & 296.62ms \\
    \hline cudaMemcpy         & 30.429ms \\
    \hline cudaMallocManaged  & 20.848ms \\
    \hline cudaEventSynchroni & 16.806ms \\
    \hline cudaDeviceSynchron & 8.6555ms \\
    \hline cuDeviceGetAttribu & 1.2526ms \\
    \hline cudaLaunchKernel   & 607.89us \\
    \hline cuDeviceTotalMem   & 551.48us \\
    \hline cudaEventRecord    & 321.05us \\ \hline
  \end{tabular} 
  \caption{The total time for all API calls
  which took longer than 200 us.}
\end{table}

\end{document}
