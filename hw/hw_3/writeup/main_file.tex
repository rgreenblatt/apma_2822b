\documentclass{article}
\usepackage[utf8]{inputenc}
\usepackage{amsmath}
\usepackage{amssymb}
\usepackage{amssymb}
\usepackage{upgreek}
\usepackage[colorlinks = true,
            linkcolor = black,
            urlcolor  = blue]{hyperref}

\usepackage[margin=1.5in]{geometry}
\usepackage{relsize}
\usepackage{color}

\title{APMA 2822b Homework 2}
\author{Ryan Greenblatt}
\date{February 2019}

\begin{document}

\setlength\parindent{0pt}

\renewcommand{\thesubsection}{\alph{subsection}}

\maketitle

\section{}

Code is attached in the email.

\section{}


The total number of FLOPS for Matrix-Matrix multiplication with matrices of size $n$ is $2 n^3$. The 
total data in bytes that must be transfered (assuming the use of doubles) is 
$8 * 3 n^2 = 24 n^2$. The arithmetic intensity for this task is 
$\frac{2n^3}{24n^2} = \frac1{12} n = \frac{4096}{12} = 341.3$. This is well above the ratio of
peak FLOPS to peak memory bandwidth for this system $(\frac{300}{55} = 5.5)$. Thus, the 
computation can be expected to ideally be FLOP bounded. However, poor cache utilization
or other bottlenecks could result in the computation not being FLOP bounded.

\section{}

I implemented an algorithm which utilizes 3 levels of blocking. The outer most level 
of blocking is split between threads and loops over the first dimension of the A 
and C matrices.  The middle level loops over the last dimension of the A matrix 
and the first dimension of the B matrix.  The inner level loops over the last dimensions 
of the C and B matrices. In addition to tuning the block size parameters, I tested 
aligning threads side by side on the outer level instead of the 3rd level of blocking. 
This greatly reduced performance. I also tested transposing the blocks of the A 
matrix and copying the blocks from the B matrix to allow for the data access to 
be more sequential during the final operation; however, this reduced performance.  
The Eigen linear algebra library was also benchmarked for comparison (by default 
the code is run without benchmarking Eigen).  I tested with and without the '-march=native' 
compiler flag to determine how much of a speed up each method is obtaining from 
the use of optimized SIMD instructions. The naive approach appears to obtain no 
benefit from SIMD, the optimized algorithm obtains a sizeable benefit from SIMD, 
and Eigen gains an even larger benefit from SIMD.

\begin{table}[]
\begin{centering}
\begin{tabular}{|l|l|l|l|}
\hline
Implementation & n    & Seconds   & GLOPS   \\ \hline
Naive          & 512  & 0.0249    & 10.5    \\ \hline
Optimized      & 512  & 0.00427   & 60.0    \\ \hline
Eigen          & 512  & 0.00333   & 75.1    \\ \hline
Naive          & 1024 & 0.181     & 11.0    \\ \hline
Optimized      & 1024 & 0.0335    & 59.6    \\ \hline
Eigen          & 1024 & 0.0127    & 158.1   \\ \hline
Naive          & 2048 & 11.9      & 1.3     \\ \hline
Optimized      & 2048 & 0.338     & 47.4    \\ \hline
Eigen          & 2048 & 0.0805    & 198.7   \\ \hline
\end{tabular}
\caption{Performance with the '-march=native' compiler argument}
\end{centering}
\end{table}


\begin{table}[]
\begin{centering}
\begin{tabular}{|l|l|l|l|}
\hline
Implementation & n    & Seconds   & GLOPS   \\ \hline
Naive          & 512  & 0.0244    & 10.2    \\ \hline
Optimized      & 512  & 0.00797   & 31.4    \\ \hline
Eigen          & 512  & 0.00403   & 62.0    \\ \hline
Naive          & 1024 & 0.221     & 9.04    \\ \hline
Optimized      & 1024 & 0.0642    & 31.1    \\ \hline
Eigen          & 1024 & 0.0199    & 100.6   \\ \hline
Naive          & 2048 & 12.2      & 1.3     \\ \hline
Optimized      & 2048 & 0.675     & 23.7    \\ \hline
Eigen          & 2048 & 0.243     & 65.9    \\ \hline
\end{tabular}
\caption{Performance without '-march=native'}
\end{centering}
\end{table}

\end{document}

