\documentclass{article}
\usepackage[utf8]{inputenc}
\usepackage{amsmath}
\usepackage{amssymb}
\usepackage{amssymb}
\usepackage{upgreek}
\usepackage[colorlinks = true,
            linkcolor = black,
            urlcolor  = blue]{hyperref}

\usepackage[margin=1.5in]{geometry}
\usepackage{relsize}
\usepackage{color}

\title{APMA 2822b Homework 1}
\author{Ryan Greenblatt}
\date{February 2019}

\begin{document}

\setlength\parindent{0pt}

\renewcommand{\thesubsection}{\alph{subsection}}

\maketitle

\stepcounter{section}

\section{}

The CPU has a peak FLOP rate of 500 GFLOPS\footnote{Within this report, I am assuming
1 GFLOPS is $1024^3$ FLOPS.} and 
an achievable bandwidth of 100 GiB/s. Thus, memory will
be the bottleneck unless the task has an operational intensity of 5 FLOPS per byte or higher.

\subsection{}

For task a, each iteration requires 3 FLOPS (the compiler may optimize it to just 2 FLOPS) and 3 doubles to
be transfered.
Each double is 8 bytes, so the FLOPS per byte is only $\frac{3}{24} = \frac{1}{8} = 0.125$. The task will 
be memory limited and
the total amount of memory transfer is $\frac{512*1024*1024}{8} * 8 * 3$ which is 1.5 GiB.
It will take $\frac{3}{200} = 0.015$ seconds. The total number of FLOPS is $\frac{512*1024*1024}{8} * 3$ and the
FLOP rate  will be $\frac{25}{2} = 12.5$ GFLOPS.


\subsection{}

For task b, each iteration requires 6 FLOPS and 5 doubles.
Each double is 8 bytes, so the FLOPS per byte is only $\frac{3}{20} = 0.15$. The task will be memory 
limited and
the total amount of memory transfer is $\frac{512*1024*1024}{8} * 8 * 5$ which is 2.5 GiB.
It will take $\frac{5}{200} = \frac{1}{40} = 0.025$ seconds. The total number of FLOPS 
is $\frac{512*1024*1024}{8} * 6$ and the
FLOP rate will be $15$ GFLOPS.

\section{}

I ran my code on the CCV using the class reserved node.
Both tasks were memory limited for both sizes. Running the tasks with reduced size increased speed
substantially because the data could remain entirely in the cache.

\begin{table}[h!]
\begin{tabular}{|l|l|l|l|}
\hline
Task & N                                      & Seconds  & GiB/s \\ \hline
a    & $\frac{512*1024*1024}8$ & 0.0368   & 27   \\ \hline
b    & $\frac{512*1024*1024}8$ & 0.0651   & 23   \\ \hline
a    & 1024                                   & 1.70e-07 & 90   \\ \hline
b    & 1024                                   & 3.52e-07 & 65   \\ \hline
\end{tabular}
\end{table}


\end{document}

